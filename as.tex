\documentclass{beamer}

% Theme
\usetheme{Madrid}
\usecolortheme{default}
\usepackage{tabularx}
%\usepackage{titling}
\definecolor{blue}{RGB}{0, 102, 204}
\setbeamercolor{structure}{fg=blue}

% Title page
\title{Statistics on further studies of students in IIT H}
\subtitle{MA4240  Applied Statistics}
\author{Shreyas Wankhede, Pradeep Mundlik, Akshitha Kola, Dhatri Reddy, Mouktika Cherukupalli, Avinash Malothu, Sumeeth Kumar, Sathvika Marri}
\institute{Indian Institute of Technology Hyderabad}
\date{\today}

% Begin document
\begin{document}

% Title slide
\begin{frame}
  \titlepage
\end{frame}

% Outline slide
\begin{frame}
  \frametitle{Outline}
  \tableofcontents
\end{frame}

% Content slides
\section{Introduction}
\begin{frame}
  \frametitle{Introduction}
  \begin{block}{}  
  This project is based on further studies of students studying at IITH. We have used statisics to deduce few conclusions from the given data, assuming that the data is random sample population.  
  \end{block}

  \begin{block}{}  
  We used sampling, more specifically volunteer sampling for collection of data through mail from students at IITH but only a few of them volunteered to respond. The data was collected from 114 students, and it is diverse with data from different years of UG.
  \end{block}

\end{frame}

\begin{frame}
  \begin{block}{Variables of interest}
      \begin{enumerate}
          \item Which department?
          \item Gender?
          \item Annual family income
          \item CGPA?
          \item Interested in further studies (if yes) 
          \begin{itemize}
              \item After how many years of work experience?
              \item Which degree?( Masters/ PhD/ MBA)
              \item Location?
              \item In which department(Same as bachelors or different?)
          \end{itemize}
          \item Interested in further studies(if no)
          \begin{itemize}
              \item Are you interested in civils?
              \item Are you interested in software development?
          \end{itemize}
      \end{enumerate}
  \end{block}  
\end{frame}

\section{Data Visualization}
\begin{frame}
  \frametitle{Data visualization}
  
\end{frame}

\section{Hypothesis Testing}
\begin{frame}
  \frametitle{Hypothesis Testing}
  \begin{block}{\textbf{Case 1:}{ Comparing CGPA of students who are willing to pursue higher studies with students who don't want to pursue  }}
     We assume our null hypothesis to be that average CGPA of students willing to go for higher studies is  greater than those who dont want to. Let $\alpha = 0.05$.
     \begin{enumerate*}
         \item  $\bar{x_1}$ = sample mean of CGPA of people willing to go for higher studies\\
         \item  $\bar{x_2}$ = sample mean of CGPA of people who  don't want to go for higher studies\\
         \item  $s^2_1$ = sample standard deviation of CGPA of people willing for higher studies\\
         \item  $s^2_2$ = sample standard deviation of CGPA of people who  don't want \\
     \end{enumerate*}
   For Hypothesis Testing we make the following statements:
      \begin{align*}
          H_0=\mu_1-\mu_2\geq0\\
          H_a=\mu1-\mu2<0
      \end{align*}
        \end{block}
    \end{frame}
    \begin{frame}{Hypothesis testing}
        \begin{block}{\textbf{Case 1 continued:}}
            Information:
             \begin{align}
                 &\bar{x_1}=8.4447\\
                 &\bar{x_2}=8.3919\\
                 &s^2_1=0.7694\\
                 &s^2_2=0.5672\\
                 &n_1=87\\
                 &n_2=26
             \end{align}
       since $\dfrac{s^2_1}{s^2_2}<4$ , we can assume the population variances would be equal. \\
       Degrees of freedom, $$dof=n_1+n_2-2=111$$
        \end{block}
    \end{frame}
    \begin{frame}{Hypothesis testing}
    \begin{block}{Case 1 continued:}
        The pooled varience will be:
           \begin{align}
               s^2_p=\dfrac{(n_1-1)s^2_1+(n_2-1)s^2_2}{n_1+n_2-2}=0.5317
           \end{align}
           The test statistic is given by:
           \begin{align}
               t=\dfrac{\bar{x_1}-\bar{x_2}-0}{s_p\sqrt{\dfrac{n_2+n_1}{n_1n_2}}}=0.3314
           \end{align}
           Using the rejection region approach, we reject $H_0$ if $t_{0.05,111}\geq -t$ where $t_{0.05,111}=-1.6587$. we have enough statistical evidence to reject null hypothesis since observed t is lesser than 1.6587
           
    \end{block}
        
    \end{frame}

\begin{frame}
  \frametitle{Hypothesis Testing}
  \begin{block}{\textbf{Case 2:}{ Hypothesized testing if students want an average work experience of more than 1 year before going for further studies }}
     Let us assume the null hypothesis as the average work experience of students willing to go for higher studies is  less than 1 year. Let $\alpha = 0.05$.

     The hypotheses are:
      \begin{align*}
          H_0=\mu\leq \mu_0\\
          H_a=\mu > \mu_0
      \end{align*}
      where $\mu_0 = 1$

     \begin{enumerate*}
         \item  $\bar{X}$ = Average years of work experience before going for further studies\\
         \item  $S^2$ = Sample standard deviation of number of years of work experience of students willing to go for higher studies\\
         \item $n$ = Number of students planning to go for further studies\\
     \end{enumerate*}
        \end{block}
    \end{frame}

    \begin{frame}{Hypothesis testing}
        \begin{block}{\textbf{Case 2 continued:}}
            Information:
             \begin{align}
                 &\bar{X}= 1.33\\
                 &S^2 = 1.74\\
                 &n=87\\
                 &df = n-1 = 86
             \end{align}
             The test static is given by:
             \begin{align}
                 t^* &= \frac{\bar{X} - \mu_0}{ S/ \sqrt{n}}
                 = \frac{1.33 - 1}{\frac{\sqrt{1.74}}{\sqrt{87}}}
                 = 2.34
             \end{align}

             Using the rejection region approach, we fail reject $H_0$ if $t_{0.05,86}\geq t$ where $t_{0.05,86} = 1.6628$. Since, $t^*$ does not fall within the region, we fail to reject $H_0$.
        \end{block}
    \end{frame}

    \begin{frame}{Hypothesis testing}
    
  \begin{block}{\textbf{Case 4:} Hypothesized proportion testing if there is enough evidence that the proportions of people opting for masters, MBA, Phd are not all equal}
    Sample data :
    \begin{tabular}{|c|c|c|c|}
        \hline
        Masters & MBA & PhD & Total\\
        \hline
        50 & 24 & 13 & 87 \\
        \hline
    \end{tabular}\\

    Let $P_{Ms}$, $P_{MBA}$, $P_{PhD}$ denote proportions of students willing to pursue Masters, MBA, PhD for higher studies \\

    $H_{0}$ :  $P_{Ms} = P_{MBA} = P_{PhD} = \dfrac{1}{3}$ \space \space \space \space $H_{a}$ : atleast one $P \neq \dfrac{1}{3}$\space \space \space \space $\alpha =0.05$\\

    Also,
    \begin{equation}
        E = \dfrac{1}{3} x 87 = 29
    \end{equation}
    and,
    \begin{equation}
        \chi^{2} = \sum\dfrac{(O - E)^{2}}{E}
    \end{equation}  
    \end{block}
\end{frame}

%\section{Hypothesis Testing}
\begin{frame}
  \frametitle{Hypothesis Testing}
  \begin{block}{\textbf{Case 4 continued:}}
   
    \begin{equation}
        \begin{aligned}
            \chi^{2} &= \dfrac{(50 - 29)^{2}}{29} + \dfrac{(24 -29)^{2}}{29} + \dfrac{(13 -29)^{2}}{29}\\
            &= 15.2 + 0.862 + 8.827\\
            &= 24.889 
        \end{aligned}
    \end{equation}
    At $df = 3 - 1 = 2 $ , p value $= 0.0001  $\\
    p value $< \alpha = 0.05$\\
    Hence there is enough evidence that population proportions are not all equal.


    \end{block}
\end{frame}

% End document
\end{document}
